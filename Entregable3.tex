% Options for packages loaded elsewhere
\PassOptionsToPackage{unicode}{hyperref}
\PassOptionsToPackage{hyphens}{url}
%
\documentclass[
]{article}
\usepackage{amsmath,amssymb}
\usepackage{lmodern}
\usepackage{iftex}
\ifPDFTeX
  \usepackage[T1]{fontenc}
  \usepackage[utf8]{inputenc}
  \usepackage{textcomp} % provide euro and other symbols
\else % if luatex or xetex
  \usepackage{unicode-math}
  \defaultfontfeatures{Scale=MatchLowercase}
  \defaultfontfeatures[\rmfamily]{Ligatures=TeX,Scale=1}
\fi
% Use upquote if available, for straight quotes in verbatim environments
\IfFileExists{upquote.sty}{\usepackage{upquote}}{}
\IfFileExists{microtype.sty}{% use microtype if available
  \usepackage[]{microtype}
  \UseMicrotypeSet[protrusion]{basicmath} % disable protrusion for tt fonts
}{}
\makeatletter
\@ifundefined{KOMAClassName}{% if non-KOMA class
  \IfFileExists{parskip.sty}{%
    \usepackage{parskip}
  }{% else
    \setlength{\parindent}{0pt}
    \setlength{\parskip}{6pt plus 2pt minus 1pt}}
}{% if KOMA class
  \KOMAoptions{parskip=half}}
\makeatother
\usepackage{xcolor}
\usepackage[margin=1in]{geometry}
\usepackage{color}
\usepackage{fancyvrb}
\newcommand{\VerbBar}{|}
\newcommand{\VERB}{\Verb[commandchars=\\\{\}]}
\DefineVerbatimEnvironment{Highlighting}{Verbatim}{commandchars=\\\{\}}
% Add ',fontsize=\small' for more characters per line
\usepackage{framed}
\definecolor{shadecolor}{RGB}{248,248,248}
\newenvironment{Shaded}{\begin{snugshade}}{\end{snugshade}}
\newcommand{\AlertTok}[1]{\textcolor[rgb]{0.94,0.16,0.16}{#1}}
\newcommand{\AnnotationTok}[1]{\textcolor[rgb]{0.56,0.35,0.01}{\textbf{\textit{#1}}}}
\newcommand{\AttributeTok}[1]{\textcolor[rgb]{0.77,0.63,0.00}{#1}}
\newcommand{\BaseNTok}[1]{\textcolor[rgb]{0.00,0.00,0.81}{#1}}
\newcommand{\BuiltInTok}[1]{#1}
\newcommand{\CharTok}[1]{\textcolor[rgb]{0.31,0.60,0.02}{#1}}
\newcommand{\CommentTok}[1]{\textcolor[rgb]{0.56,0.35,0.01}{\textit{#1}}}
\newcommand{\CommentVarTok}[1]{\textcolor[rgb]{0.56,0.35,0.01}{\textbf{\textit{#1}}}}
\newcommand{\ConstantTok}[1]{\textcolor[rgb]{0.00,0.00,0.00}{#1}}
\newcommand{\ControlFlowTok}[1]{\textcolor[rgb]{0.13,0.29,0.53}{\textbf{#1}}}
\newcommand{\DataTypeTok}[1]{\textcolor[rgb]{0.13,0.29,0.53}{#1}}
\newcommand{\DecValTok}[1]{\textcolor[rgb]{0.00,0.00,0.81}{#1}}
\newcommand{\DocumentationTok}[1]{\textcolor[rgb]{0.56,0.35,0.01}{\textbf{\textit{#1}}}}
\newcommand{\ErrorTok}[1]{\textcolor[rgb]{0.64,0.00,0.00}{\textbf{#1}}}
\newcommand{\ExtensionTok}[1]{#1}
\newcommand{\FloatTok}[1]{\textcolor[rgb]{0.00,0.00,0.81}{#1}}
\newcommand{\FunctionTok}[1]{\textcolor[rgb]{0.00,0.00,0.00}{#1}}
\newcommand{\ImportTok}[1]{#1}
\newcommand{\InformationTok}[1]{\textcolor[rgb]{0.56,0.35,0.01}{\textbf{\textit{#1}}}}
\newcommand{\KeywordTok}[1]{\textcolor[rgb]{0.13,0.29,0.53}{\textbf{#1}}}
\newcommand{\NormalTok}[1]{#1}
\newcommand{\OperatorTok}[1]{\textcolor[rgb]{0.81,0.36,0.00}{\textbf{#1}}}
\newcommand{\OtherTok}[1]{\textcolor[rgb]{0.56,0.35,0.01}{#1}}
\newcommand{\PreprocessorTok}[1]{\textcolor[rgb]{0.56,0.35,0.01}{\textit{#1}}}
\newcommand{\RegionMarkerTok}[1]{#1}
\newcommand{\SpecialCharTok}[1]{\textcolor[rgb]{0.00,0.00,0.00}{#1}}
\newcommand{\SpecialStringTok}[1]{\textcolor[rgb]{0.31,0.60,0.02}{#1}}
\newcommand{\StringTok}[1]{\textcolor[rgb]{0.31,0.60,0.02}{#1}}
\newcommand{\VariableTok}[1]{\textcolor[rgb]{0.00,0.00,0.00}{#1}}
\newcommand{\VerbatimStringTok}[1]{\textcolor[rgb]{0.31,0.60,0.02}{#1}}
\newcommand{\WarningTok}[1]{\textcolor[rgb]{0.56,0.35,0.01}{\textbf{\textit{#1}}}}
\usepackage{graphicx}
\makeatletter
\def\maxwidth{\ifdim\Gin@nat@width>\linewidth\linewidth\else\Gin@nat@width\fi}
\def\maxheight{\ifdim\Gin@nat@height>\textheight\textheight\else\Gin@nat@height\fi}
\makeatother
% Scale images if necessary, so that they will not overflow the page
% margins by default, and it is still possible to overwrite the defaults
% using explicit options in \includegraphics[width, height, ...]{}
\setkeys{Gin}{width=\maxwidth,height=\maxheight,keepaspectratio}
% Set default figure placement to htbp
\makeatletter
\def\fps@figure{htbp}
\makeatother
\setlength{\emergencystretch}{3em} % prevent overfull lines
\providecommand{\tightlist}{%
  \setlength{\itemsep}{0pt}\setlength{\parskip}{0pt}}
\setcounter{secnumdepth}{-\maxdimen} % remove section numbering
\ifLuaTeX
  \usepackage{selnolig}  % disable illegal ligatures
\fi
\IfFileExists{bookmark.sty}{\usepackage{bookmark}}{\usepackage{hyperref}}
\IfFileExists{xurl.sty}{\usepackage{xurl}}{} % add URL line breaks if available
\urlstyle{same} % disable monospaced font for URLs
\hypersetup{
  pdftitle={Entregable 3},
  pdfauthor={Mayron Andree Ortiz Tineo},
  hidelinks,
  pdfcreator={LaTeX via pandoc}}

\title{Entregable 3}
\author{Mayron Andree Ortiz Tineo}
\date{2023-05-07}

\begin{document}
\maketitle

\hypertarget{entregable-3}{%
\subsection{Entregable 3}\label{entregable-3}}

\begin{verbatim}
## 
## Attaching package: 'dplyr'
\end{verbatim}

\begin{verbatim}
## The following objects are masked from 'package:stats':
## 
##     filter, lag
\end{verbatim}

\begin{verbatim}
## The following objects are masked from 'package:base':
## 
##     intersect, setdiff, setequal, union
\end{verbatim}

\begin{verbatim}
## 
## Attaching package: 'mltools'
\end{verbatim}

\begin{verbatim}
## The following object is masked from 'package:tidyr':
## 
##     replace_na
\end{verbatim}

\begin{verbatim}
## 
## Attaching package: 'data.table'
\end{verbatim}

\begin{verbatim}
## The following objects are masked from 'package:dplyr':
## 
##     between, first, last
\end{verbatim}

\hypertarget{pregunta1}{%
\subsection{Pregunta1}\label{pregunta1}}

\begin{Shaded}
\begin{Highlighting}[]
\CommentTok{\# Importar el archivo}
\NormalTok{http\_data }\OtherTok{\textless{}{-}} \FunctionTok{read\_table}\NormalTok{(}\StringTok{"C:}\SpecialCharTok{\textbackslash{}\textbackslash{}}\StringTok{Users}\SpecialCharTok{\textbackslash{}\textbackslash{}}\StringTok{Andree}\SpecialCharTok{\textbackslash{}\textbackslash{}}\StringTok{Documents}\SpecialCharTok{\textbackslash{}\textbackslash{}}\StringTok{epa{-}http.csv"}\NormalTok{, }\AttributeTok{col\_names =} \ConstantTok{FALSE}\NormalTok{)}
\end{Highlighting}
\end{Shaded}

\begin{verbatim}
## 
## -- Column specification --------------------------------------------------------
## cols(
##   X1 = col_character(),
##   X2 = col_character(),
##   X3 = col_character(),
##   X4 = col_character(),
##   X5 = col_character(),
##   X6 = col_double(),
##   X7 = col_character()
## )
\end{verbatim}

\begin{Shaded}
\begin{Highlighting}[]
\CommentTok{\# editar encabezados de las columnas}
\FunctionTok{colnames}\NormalTok{(http\_data) }\OtherTok{\textless{}{-}} \FunctionTok{c}\NormalTok{(}\StringTok{"Direction"}\NormalTok{, }\StringTok{"Date"}\NormalTok{, }\StringTok{"Method"}\NormalTok{, }\StringTok{"Resource"}\NormalTok{, }\StringTok{"Protocol"}\NormalTok{, }\StringTok{"Responsecode"}\NormalTok{, }\StringTok{"Byte"}\NormalTok{)}

\CommentTok{\#Pregunta 1}
\CommentTok{\# Conversiones}
\NormalTok{http\_data}\SpecialCharTok{$}\NormalTok{Method }\OtherTok{\textless{}{-}} \FunctionTok{as.factor}\NormalTok{(http\_data}\SpecialCharTok{$}\NormalTok{Method)}
\NormalTok{http\_data}\SpecialCharTok{$}\NormalTok{Protocol }\OtherTok{\textless{}{-}} \FunctionTok{as.factor}\NormalTok{(http\_data}\SpecialCharTok{$}\NormalTok{Protocol)}
\NormalTok{http\_data}\SpecialCharTok{$}\NormalTok{Responsecode }\OtherTok{\textless{}{-}} \FunctionTok{as.factor}\NormalTok{(http\_data}\SpecialCharTok{$}\NormalTok{Responsecode)}
\NormalTok{http\_data}\SpecialCharTok{$}\NormalTok{Byte }\OtherTok{\textless{}{-}} \FunctionTok{as.numeric}\NormalTok{(http\_data}\SpecialCharTok{$}\NormalTok{Byte)}
\CommentTok{\#http\_data$Resource \textless{}{-} as.factor(http\_data$Resource)}
\CommentTok{\# Conversion de valores NA por 0}
\NormalTok{http\_data}\SpecialCharTok{$}\NormalTok{Byte }\OtherTok{\textless{}{-}} \FunctionTok{ifelse}\NormalTok{(}\FunctionTok{is.na}\NormalTok{(http\_data}\SpecialCharTok{$}\NormalTok{Byte), }\DecValTok{0}\NormalTok{, http\_data}\SpecialCharTok{$}\NormalTok{Byte)}
\FunctionTok{nrow}\NormalTok{(http\_data)}
\end{Highlighting}
\end{Shaded}

\begin{verbatim}
## [1] 47748
\end{verbatim}

\begin{Shaded}
\begin{Highlighting}[]
\FunctionTok{View}\NormalTok{(http\_data)}
\end{Highlighting}
\end{Shaded}

\hypertarget{pregunta2}{%
\subsection{Pregunta2}\label{pregunta2}}

\begin{Shaded}
\begin{Highlighting}[]
\CommentTok{\#Pregunta 2}
\CommentTok{\# Crear una tabla con las direcciones repetidas y así direcciones únicas}
\NormalTok{direction\_table }\OtherTok{\textless{}{-}} \FunctionTok{data.frame}\NormalTok{(}\AttributeTok{Direction =}\NormalTok{ http\_data}\SpecialCharTok{$}\NormalTok{Direction, }\AttributeTok{Responsecode =}\NormalTok{http\_data}\SpecialCharTok{$}\NormalTok{Responsecode) }
\NormalTok{concurrences }\OtherTok{\textless{}{-}} \FunctionTok{as.data.frame}\NormalTok{(}\FunctionTok{table}\NormalTok{(direction\_table))}

\CommentTok{\# Filtrar valores existentes y ordenarlo ascendente por la columna Responsecode}
\CommentTok{\# 200, 302, 304, 400, 403, 404, 500, 501}
\NormalTok{direction\_data }\OtherTok{\textless{}{-}} \FunctionTok{filter}\NormalTok{(concurrences, Freq }\SpecialCharTok{\textgreater{}} \DecValTok{0}\NormalTok{) }


\NormalTok{direction\_data }\OtherTok{\textless{}{-}}\NormalTok{ direction\_data }\SpecialCharTok{\%\textgreater{}\%}
  \FunctionTok{arrange}\NormalTok{(Responsecode)}
\FunctionTok{View}\NormalTok{(direction\_data)}


\NormalTok{code200\_data }\OtherTok{\textless{}{-}}\NormalTok{ direction\_data }\SpecialCharTok{\%\textgreater{}\%} \FunctionTok{filter}\NormalTok{(Responsecode }\SpecialCharTok{==} \DecValTok{200}\NormalTok{)}
\FunctionTok{nrow}\NormalTok{(code200\_data)}
\end{Highlighting}
\end{Shaded}

\begin{verbatim}
## [1] 2296
\end{verbatim}

\begin{Shaded}
\begin{Highlighting}[]
\NormalTok{code302\_data }\OtherTok{\textless{}{-}}\NormalTok{ direction\_data }\SpecialCharTok{\%\textgreater{}\%} \FunctionTok{filter}\NormalTok{(Responsecode }\SpecialCharTok{==} \DecValTok{302}\NormalTok{)}
\FunctionTok{nrow}\NormalTok{(code302\_data)}
\end{Highlighting}
\end{Shaded}

\begin{verbatim}
## [1] 970
\end{verbatim}

\begin{Shaded}
\begin{Highlighting}[]
\NormalTok{code304\_data }\OtherTok{\textless{}{-}}\NormalTok{ direction\_data }\SpecialCharTok{\%\textgreater{}\%} \FunctionTok{filter}\NormalTok{(Responsecode }\SpecialCharTok{==} \DecValTok{304}\NormalTok{)}
\FunctionTok{nrow}\NormalTok{(code304\_data)}
\end{Highlighting}
\end{Shaded}

\begin{verbatim}
## [1] 505
\end{verbatim}

\begin{Shaded}
\begin{Highlighting}[]
\NormalTok{code400\_data }\OtherTok{\textless{}{-}}\NormalTok{ direction\_data }\SpecialCharTok{\%\textgreater{}\%} \FunctionTok{filter}\NormalTok{(Responsecode }\SpecialCharTok{==} \DecValTok{400}\NormalTok{)}
\FunctionTok{nrow}\NormalTok{(code400\_data)}
\end{Highlighting}
\end{Shaded}

\begin{verbatim}
## [1] 1
\end{verbatim}

\begin{Shaded}
\begin{Highlighting}[]
\NormalTok{code403\_data }\OtherTok{\textless{}{-}}\NormalTok{ direction\_data }\SpecialCharTok{\%\textgreater{}\%} \FunctionTok{filter}\NormalTok{(Responsecode }\SpecialCharTok{==} \DecValTok{403}\NormalTok{)}
\FunctionTok{nrow}\NormalTok{(code403\_data)}
\end{Highlighting}
\end{Shaded}

\begin{verbatim}
## [1] 5
\end{verbatim}

\begin{Shaded}
\begin{Highlighting}[]
\NormalTok{code404\_data }\OtherTok{\textless{}{-}}\NormalTok{ direction\_data }\SpecialCharTok{\%\textgreater{}\%} \FunctionTok{filter}\NormalTok{(Responsecode }\SpecialCharTok{==} \DecValTok{404}\NormalTok{)}
\FunctionTok{nrow}\NormalTok{(code404\_data)}
\end{Highlighting}
\end{Shaded}

\begin{verbatim}
## [1] 152
\end{verbatim}

\begin{Shaded}
\begin{Highlighting}[]
\NormalTok{code500\_data }\OtherTok{\textless{}{-}}\NormalTok{ direction\_data }\SpecialCharTok{\%\textgreater{}\%} \FunctionTok{filter}\NormalTok{(Responsecode }\SpecialCharTok{==} \DecValTok{500}\NormalTok{)}
\FunctionTok{nrow}\NormalTok{(code500\_data)}
\end{Highlighting}
\end{Shaded}

\begin{verbatim}
## [1] 29
\end{verbatim}

\begin{Shaded}
\begin{Highlighting}[]
\NormalTok{code501\_data }\OtherTok{\textless{}{-}}\NormalTok{ direction\_data }\SpecialCharTok{\%\textgreater{}\%} \FunctionTok{filter}\NormalTok{(Responsecode }\SpecialCharTok{==} \DecValTok{501}\NormalTok{)}
\FunctionTok{nrow}\NormalTok{(code501\_data)}
\end{Highlighting}
\end{Shaded}

\begin{verbatim}
## [1] 11
\end{verbatim}

\hypertarget{pregunta3}{%
\subsection{Pregunta3}\label{pregunta3}}

\begin{Shaded}
\begin{Highlighting}[]
\CommentTok{\#Pregunta 3}

\CommentTok{\# contar la frecuencia}
\NormalTok{freq\_http }\OtherTok{\textless{}{-}} \FunctionTok{table}\NormalTok{(http\_data}\SpecialCharTok{$}\NormalTok{Method)}
\NormalTok{method\_data }\OtherTok{\textless{}{-}} \FunctionTok{data.frame}\NormalTok{(}\AttributeTok{http =} \FunctionTok{names}\NormalTok{(freq\_http), }\AttributeTok{freq\_http =} \FunctionTok{as.vector}\NormalTok{(freq\_http))}
\NormalTok{method\_data}
\end{Highlighting}
\end{Shaded}

\begin{verbatim}
##    http freq_http
## 1  "GET     46020
## 2 "HEAD       106
## 3 "POST      1622
\end{verbatim}

\begin{Shaded}
\begin{Highlighting}[]
\CommentTok{\# Hallar la frecuencia de la columna http, filtrar los tipos de imagen}
\NormalTok{different\_image\_data }\OtherTok{\textless{}{-}}\NormalTok{ http\_data }\SpecialCharTok{\%\textgreater{}\%}
  \FunctionTok{filter}\NormalTok{(}\SpecialCharTok{!}\FunctionTok{grepl}\NormalTok{(}\StringTok{"(?i)}\SpecialCharTok{\textbackslash{}\textbackslash{}}\StringTok{.(gif|jpg|jpeg|png|bmp)$"}\NormalTok{, Resource))}

\NormalTok{freq\_http2 }\OtherTok{\textless{}{-}} \FunctionTok{table}\NormalTok{(different\_image\_data}\SpecialCharTok{$}\NormalTok{Method)}
\NormalTok{method2\_data }\OtherTok{\textless{}{-}} \FunctionTok{data.frame}\NormalTok{(}\AttributeTok{http =} \FunctionTok{names}\NormalTok{(freq\_http2), }\AttributeTok{freq\_http2 =} \FunctionTok{as.vector}\NormalTok{(freq\_http2))}

\NormalTok{method\_data}
\end{Highlighting}
\end{Shaded}

\begin{verbatim}
##    http freq_http
## 1  "GET     46020
## 2 "HEAD       106
## 3 "POST      1622
\end{verbatim}

\hypertarget{pregunta4}{%
\subsection{Pregunta4}\label{pregunta4}}

\begin{Shaded}
\begin{Highlighting}[]
\CommentTok{\#Pregunta 4}
\NormalTok{tabla\_frecuencia }\OtherTok{\textless{}{-}}\FunctionTok{table}\NormalTok{(http\_data}\SpecialCharTok{$}\NormalTok{Responsecode)}
\NormalTok{responsecode\_table }\OtherTok{\textless{}{-}} \FunctionTok{data.frame}\NormalTok{(}\AttributeTok{Responsecode =} \FunctionTok{names}\NormalTok{(tabla\_frecuencia), }
                                  \AttributeTok{Frecuencia =} \FunctionTok{as.vector}\NormalTok{(tabla\_frecuencia))}


\FunctionTok{ggplot}\NormalTok{(responsecode\_table, }\FunctionTok{aes}\NormalTok{(}\AttributeTok{x =}\NormalTok{ Responsecode, }\AttributeTok{y =}\NormalTok{ Frecuencia,  }\AttributeTok{fill =}\NormalTok{ Responsecode)) }\SpecialCharTok{+}
  \FunctionTok{geom\_bar}\NormalTok{(}\AttributeTok{stat =} \StringTok{"identity"}\NormalTok{) }\SpecialCharTok{+}
  \FunctionTok{scale\_fill\_manual}\NormalTok{(}\AttributeTok{values =} \FunctionTok{c}\NormalTok{(}\StringTok{"\#55024a"}\NormalTok{, }\StringTok{"\#9dab34"}\NormalTok{, }\StringTok{"\#e16639"}\NormalTok{,}\StringTok{"\#eb214e"}\NormalTok{, }\StringTok{"\#9ed99e"}\NormalTok{,}\StringTok{"\#9b0800"}\NormalTok{,}\StringTok{"\#82bda7"}\NormalTok{,}\StringTok{"\#f69a0b"}\NormalTok{)) }\SpecialCharTok{+}
  \FunctionTok{labs}\NormalTok{(}\AttributeTok{title =} \StringTok{"Gráfico de Respuesta"}\NormalTok{,}
       \AttributeTok{x =} \StringTok{"Respuesta"}\NormalTok{,}
       \AttributeTok{y =} \StringTok{"Frecuencia"}\NormalTok{)}
\end{Highlighting}
\end{Shaded}

\includegraphics{Entregable3_files/figure-latex/question4-1.pdf}

\begin{Shaded}
\begin{Highlighting}[]
\FunctionTok{ggplot}\NormalTok{(responsecode\_table, }\FunctionTok{aes}\NormalTok{(}\AttributeTok{x =} \StringTok{""}\NormalTok{, }\AttributeTok{y =}\NormalTok{ Frecuencia, }\AttributeTok{fill =}\NormalTok{ Responsecode)) }\SpecialCharTok{+}
  \FunctionTok{geom\_bar}\NormalTok{(}\AttributeTok{stat =} \StringTok{"identity"}\NormalTok{, }\AttributeTok{color =} \StringTok{"white"}\NormalTok{) }\SpecialCharTok{+}
  \FunctionTok{coord\_polar}\NormalTok{(}\StringTok{"y"}\NormalTok{, }\AttributeTok{start =} \DecValTok{0}\NormalTok{) }\SpecialCharTok{+}
  \FunctionTok{labs}\NormalTok{(}\AttributeTok{title =} \StringTok{"Gráfico de Respuesta"}\NormalTok{,}
       \AttributeTok{fill =} \StringTok{"respuesta"}\NormalTok{) }\SpecialCharTok{+}
  \FunctionTok{theme\_void}\NormalTok{()}
\end{Highlighting}
\end{Shaded}

\includegraphics{Entregable3_files/figure-latex/question4-2.pdf}

\hypertarget{pregunta5}{%
\subsection{Pregunta5}\label{pregunta5}}

\begin{Shaded}
\begin{Highlighting}[]
\CommentTok{\#Pregunta 5}


\NormalTok{http\_data\_filtered }\OtherTok{\textless{}{-}}\NormalTok{ http\_data[, }\FunctionTok{c}\NormalTok{(}\StringTok{"Method"}\NormalTok{, }\StringTok{"Responsecode"}\NormalTok{, }\StringTok{"Protocol"}\NormalTok{)]}

\NormalTok{epa\_http\_one\_hot }\OtherTok{\textless{}{-}} \FunctionTok{one\_hot}\NormalTok{(}\FunctionTok{as.data.table}\NormalTok{(http\_data\_filtered), }\AttributeTok{sparsifyNAs =} \ConstantTok{TRUE}\NormalTok{)}

\NormalTok{http\_data}\SpecialCharTok{$}\NormalTok{Resource\_size }\OtherTok{\textless{}{-}} \FunctionTok{nchar}\NormalTok{(http\_data}\SpecialCharTok{$}\NormalTok{Resource)}

\CommentTok{\# Agrupar 4 y 3}

\NormalTok{results2 }\OtherTok{\textless{}{-}} \FunctionTok{kmeans}\NormalTok{(epa\_http\_one\_hot, }\AttributeTok{centers =} \DecValTok{4}\NormalTok{)}

\NormalTok{results3 }\OtherTok{\textless{}{-}} \FunctionTok{kmeans}\NormalTok{(epa\_http\_one\_hot, }\AttributeTok{centers =} \DecValTok{3}\NormalTok{)}
\end{Highlighting}
\end{Shaded}

\hypertarget{pregunta6}{%
\subsection{Pregunta6}\label{pregunta6}}

\begin{Shaded}
\begin{Highlighting}[]
\CommentTok{\# Pregunta 6}

\CommentTok{\# Gráfico en base a la columna byte y Resource\_size según el tipo de agrupamiento}

\FunctionTok{set.seed}\NormalTok{(}\DecValTok{123}\NormalTok{) }

\CommentTok{\# Gráfica con cluster 4}
\CommentTok{\# Se usará colores aleatorios}
\CommentTok{\#colores2 \textless{}{-} rainbow(n = length(unique(results2$cluster)))}
\NormalTok{colores2 }\OtherTok{\textless{}{-}} \FunctionTok{c}\NormalTok{(}\StringTok{"\#f06b50"}\NormalTok{, }\StringTok{"\#8cbfaf"}\NormalTok{, }\StringTok{"\#fca699"}\NormalTok{, }\StringTok{"\#91204d"}\NormalTok{)}
\NormalTok{grap1 }\OtherTok{\textless{}{-}} \FunctionTok{plot}\NormalTok{(}\AttributeTok{x =}\NormalTok{ http\_data}\SpecialCharTok{$}\NormalTok{Byte, }\AttributeTok{y =}\NormalTok{ http\_data}\SpecialCharTok{$}\NormalTok{Resource\_size, }\AttributeTok{col =}\NormalTok{ colores2[results2}\SpecialCharTok{$}\NormalTok{cluster], }\AttributeTok{main=}\StringTok{"Gráfico de 4 clusters"}\NormalTok{)}
\CommentTok{\# se hará la conversión de notación cientifica a númerica}
\FunctionTok{options}\NormalTok{(}\AttributeTok{scipen =} \DecValTok{999}\NormalTok{)}
\CommentTok{\# Crear leyenda}
\FunctionTok{legend}\NormalTok{(}\StringTok{"topright"}\NormalTok{, }\AttributeTok{legend =} \FunctionTok{levels}\NormalTok{(}\FunctionTok{factor}\NormalTok{(results2}\SpecialCharTok{$}\NormalTok{cluster)), }\AttributeTok{col =}\NormalTok{ colores2, }\AttributeTok{pch =} \DecValTok{16}\NormalTok{)}
\end{Highlighting}
\end{Shaded}

\includegraphics{Entregable3_files/figure-latex/question6-1.pdf}

\begin{Shaded}
\begin{Highlighting}[]
\CommentTok{\#Gráfica de Cluster 3}

\CommentTok{\#colores3 \textless{}{-} rainbow(n = length(unique(results3$cluster)))}
\NormalTok{colores3 }\OtherTok{\textless{}{-}} \FunctionTok{c}\NormalTok{(}\StringTok{"\#ad2bad"}\NormalTok{,}\StringTok{"\#00988d"}\NormalTok{, }\StringTok{"\#dbbf6b"}\NormalTok{)}
\NormalTok{grap2 }\OtherTok{\textless{}{-}} \FunctionTok{plot}\NormalTok{(}\AttributeTok{x =}\NormalTok{ http\_data}\SpecialCharTok{$}\NormalTok{Byte, }\AttributeTok{y =}\NormalTok{ http\_data}\SpecialCharTok{$}\NormalTok{Resource\_size, }\AttributeTok{col =}\NormalTok{ colores3[results3}\SpecialCharTok{$}\NormalTok{cluster], }\AttributeTok{main=}\StringTok{"Gráfico de 3 clusters"}\NormalTok{)}
\FunctionTok{options}\NormalTok{(}\AttributeTok{scipen =} \DecValTok{999}\NormalTok{)}
\CommentTok{\# Creando leyenda}
\FunctionTok{legend}\NormalTok{(}\StringTok{"topright"}\NormalTok{, }\AttributeTok{legend =} \FunctionTok{levels}\NormalTok{(}\FunctionTok{factor}\NormalTok{(results3}\SpecialCharTok{$}\NormalTok{cluster)), }\AttributeTok{col =}\NormalTok{ colores3, }\AttributeTok{pch =} \DecValTok{16}\NormalTok{)}
\end{Highlighting}
\end{Shaded}

\includegraphics{Entregable3_files/figure-latex/question6-2.pdf}

\end{document}
